\documentclass[12pt,a4paper]{article}

\setlength{\textwidth}{165mm}
\setlength{\textheight}{240mm}
\setlength{\parindent}{0mm} % S{\aa} meget rykkes ind efter afsnit
\setlength{\parskip}{\parsep}
\setlength{\headheight}{0mm}
\setlength{\headsep}{0mm}
\setlength{\hoffset}{-2.5mm}
\setlength{\voffset}{0mm}
\setlength{\footskip}{15mm}
\setlength{\oddsidemargin}{0mm}
\setlength{\topmargin}{0mm}
\setlength{\evensidemargin}{0mm}


\usepackage[T1]{fontenc}
\usepackage[all]{xy}
\usepackage{graphicx}    % For grafik (billederfiler)
\usepackage[T1]{fontenc} % For at blande \textsc{} med \textbf{}
\usepackage[utf8]{inputenc}
\usepackage{amsfonts,amsmath,amssymb}
\usepackage{eucal}
\usepackage[danish]{babel}
\usepackage{enumerate}  
\usepackage{hyperref}
\usepackage{url}
\usepackage{array}
\usepackage{mathptmx}
\usepackage{amsmath}
\usepackage{multirow}
\usepackage[dvipsnames,usenames]{color}
\usepackage{tabularx,colortbl,xcolor}

\DeclareSymbolFont{usualmathcal}{OMS}{cmsy}{m}{n}
\DeclareSymbolFontAlphabet{\mathcal}{usualmathcal}



\begin{document}
\title{PKSU Delrapport 2}
\maketitle
\begin{center}
Jeppe Schönemann Skov, Rose Sofie Greve, Frederik Leed Henriksen \\ \hfill \\ \hfill \\ 
\end{center}
\newpage
\tableofcontents
\newpage
\section{Abstract}
We are making a website for a small Bed and Breakfast on Isla Margarita, Venezuela. The place is working towards becoming a sustainable and self-suffficient mini-hostel, where guests can enjoy organic greens from the garden.
The indexpage on the website will contain a description of the place and the surrounding area.
The website will contain a simple booking- and online payment system.
The website will have a link with pictures of the houses and surrounding facilities, and a link with information on the different activities taking place at the hostel.
With courtesy to the place working with ecology and sustainability, the layout of the website will be simple and inspired by the colors of nature.
Further, we will make an administrating part of the system, from where our costumer can see and edit the bookings. The administrator will also be able to edit the website photos and text descriptions. 
\newpage
\section{IT-projektets formål og rammer}
F. Skal indeholde et bookingsystem og information om hostel \\

A. Skal administrere bookninger, herunder til og afmelde folk samt finde kontaktoplysninger på de folk der har oprettet en booking.\\
 
C. Udvikling og endeligt brug af systemet skal være minded på at dem der skal betjene det ikke har stor teknisk erfaring. Vi skal derfor udvikle minded på at de tekniske detaljer ikke er deres fokus.\\

T. Systemet skal både udvikles og kunne betjenes på en billig pc med internet. Ingen yderligere krav er på nuværende tidspunkt planlagt. \\

O. Mennesker der søger at booke værelse. Organisering af hvilke værelser der er ledige og hvor længe.\\

R. Systemet skal løse administrative problematikker i henhold til at holde styr på kontaktoplysninger af kunder. Holde styr på hvilke værelser der er ledige og hvornår disse er ledige. Systemet skal også reklamere for hostel'et.\\
\newpage
\section{Kravspecifikationer for IT-løsningen}
\subsection{a}
\textbf{Funktionelle krav:} \\
	Systemet skal understøtte et bookingsystem der indeholder kontaktoplysninger om kunder 	samt muliggøre bookings af kunder og at administratoren kan booke/af-booke kunder. Desuden skal der være et betalingssystem så man kan betale online \footnote{Vi har ikke på nuværende tidspunkt tilstrækkelig indsigt i hvorledes et betalingssystem skal kunne fungere eller hvordan man man kan implementere dette til at vi har ladet det være en del af resten af de tekniske kravspecifikationer indtil vi har en dybere forståelse af dette. Det er altså et projekt vi tager fat på senere i udviklingen (men det er ikke glemt!)  } \\ \hfill \\
       \textbf{Ikke-funktionelle krav:} \\
	Systemet skal være brugervenligt i henhold til folk uden megen teknisk erfaring. 	Hjemmesiden skal indeholde billeder/information om hostelet. Siden skal være i farver og 	stil der afspejler hostelets stil og holdning til miljøet. \\ \hfill \\
      \textbf{Begrænsninger:} \\
	Siden skal kunne vises på en billig computer og uden hurtigt internet til rådighed.
\subsection{b}
Der er vedlagt et billed der illustrerer hvorledes administratoren skal logge ind. Men derved også har ret til at slette bookninger og hente information om dem der bookede et værelse. Kunden skal ikke logge ind, men har så også kun ret til at booke.\\ 
\includegraphics[scale=0.7]{useCaseModel.jpg}
\subsection{c}
\textbf{1.} Admin log-in. Her indtastes administratorens oplysninger på log-in skærmen. Dette bliver 	videreført til sikkerhedsverificering som beder beholdningen af administrator oplysninger 	om at lede efter et match. Hvis dette match findes bliver man logget in, ellers bliver man 	afvist. \\
	\textbf{2.} Admin interaktion. administratoren logger ind (trin 1) og kommer ind på 	administratorsiden. Her forespørger administrator siden om information på et rum eller at 	oprette/slette en bookning. Dette bliver sendt til validering som tjekker om informationen er 	meningsfuld (fx om man vil oprette en bookning til en dato der er passeret). Herefter bliver 	informationen sendt videre til booking databasen hvor oplysninger bliver hentet og tilbage 	forbi validering hvor man igen validere (fx om man opretter en booking på et allerede 	optaget værelse). Bliver forespørgselen på noget tidspunkt dømt invalid, vil man på 	administrator siden modtage passende fejlmeddelelse. Hvis alt går godt vil man få sin 	information eller få at vide at oprettelsen/fjernelsen af en bookning skete planmæssigt.\\
	\textbf{3.} kunde booking. Her indtaster kunden sine oplysninger på booking siden og disse 	oplysninger bliver sendt videre til validering (fx for få cifre i tlf. nummer). Passere kunden 	dette bliver forespørgslen sendt videre til booking databasen som oplyser validering om det 	ønskede værelse er ledigt for booking hvorefter validering genevaluerer. Hvis forespørgslen 	er acceptabel bliver ændringen gennemført hvorefter kunden bliver informeret om en 		succesfuld booking. Hvis validation på noget tidspunkt i processen validere til falskt, bliver 	processen stoppet og kunden bliver i stedet mødt af en fejlmeddelelse.
\subsection{d}
Siden vi endnu mangler meget erfaring inden for web-programmering, er vi ikke sikre på at følgende fremgangsmåde vil virke. Vores foreløbige ide til at lave et bookingsystem er som følger i billedet at vi laver en klasse som styre de forskellige rum i hostelet(Main). Så laver vi en klasse der holder styr på hvor mange senge hvert værelse har, samt en ”kalender” pr. værelse(Room). Så har vi en klasse der holder styr på reservationerne(booking). Derudover er der en klasse der holder styr på tidsenheder() og en der holder styr på personinformation(Customer). Dette er indskrevet med java i tankerne, men vi er som sagt ikke klar over hvordan man eksekvere kode i sin html-kode endnu, så dette er et work-in-progress.\\
\includegraphics[scale=0.4]{unfinished.jpg}
\subsection{e}
”Home” siden og ”admin log-in” siden er vores  boundary objects, da de er det første i vores system man støder på.
Vores entity objects omfatter alle vores web sider(udover dem der er boudary objects!) da de udgør en meningsfuld del af systemet. Derudover er vores ”admin credentials database” og vores ”booking database” også betragtet som entities da de indeholder al information i systemet.
Vores control objects er ”admin log-in”, ”create booking”, ”delete booking” og ”retrive info”. Dette er de fordi de ikke selv opbevare noget information, men spiller en afgørende rolle i hvad for noget infromation der bliver lagret/slettet. \\
\includegraphics[scale=0.6]{BCE.jpg}
\subsection{f}
\includegraphics[scale=0.4]{adminInteraction.jpg}\\
\includegraphics[scale=0.4]{adminLog-in.jpg}\\
\includegraphics[scale=0.4]{customerLog-in.jpg}
\newpage
\section{Systemdesign sammenfatning}
\textbf{Systemet som det er nu:}
vi har foreløbigt lavet skabelonen til hjemmesiden som den endeligt skal se ud. Dette er gjort i html og css kode. Derudover har vi implementeret links så man kan navigere mellem siderne. 

\textbf{De væsentligste mangler:}
En administrator side hvor man kan ændre billederne på siden. 
Derudover skal der også være et bookingsystem som kunder kan oprette bookninger og som administratoren via administrator siden kan oprette/slette bookninger og hente information om kunden. Desuden mangler der et betalingssystem på siden.
\newpage
\section{Program- og systemtest}
\newpage
\section{Brugergrænseflade og interaktonsdesign}
\newpage
\section{projektsamarbejdet}
\newpage
\section{Litteraturreview}
\subsection{Designing for usability}
\subsection{A rational design process}
\end{document}